\documentclass{article}

% Encoding and Fonts
\usepackage[utf8]{inputenc}
\usepackage[T1]{fontenc}
\usepackage{lmodern}

% Geometry and Layout
\usepackage{csquotes}
\usepackage[a4paper, margin=3.5cm]{geometry}
\usepackage{microtype}
\usepackage{parskip}
\usepackage[onehalfspacing]{setspace}

% Colors and Hyperlinks
\usepackage{xcolor}
\usepackage[pdfusetitle, colorlinks=true, linkcolor=blue, urlcolor=violet]{hyperref}

% Theorem-like Environments
\usepackage{csthm}

% Code Listings
\usepackage{listings}
\lstset{%
    basicstyle=\scriptsize\ttfamily,
    breaklines=true,
    frame=single,
    columns=fullflexible,
    numbers=left,
    numberstyle=\tiny\ttfamily\color{gray},
    breakindent=0pt
}

\title{\bfseries Package Documentation for \texttt{csthm}}
\author{Agni Datta}
\date{August 31, 2024}

\begin{document}

\maketitle
\tableofcontents

\section{Introduction}

The \texttt{csthm} package provides customized theorem-like environments specifically designed for computer science documents. It offers a set of pre-defined theorem styles and environments to streamline the creation of theorems, definitions, remarks, and other common structures in computer science papers and documents.

\section{Installation}

To install the \texttt{csthm} package:

\begin{enumerate}
    \item Run \texttt{tex csthm.ins} to generate \texttt{csthm.sty}
    \item Move \texttt{csthm.sty} to your TeX tree or project directory
    \item Use \verb|\usepackage{csthm}| in your LaTeX documents
\end{enumerate}

\section{Usage}

\subsection{Loading the Package}

To use the package, include it in your LaTeX document's preamble:

\begin{verbatim}
\usepackage{csthm}
\end{verbatim}

If you want to use the package with \texttt{cleveref} support:

\begin{verbatim}
\usepackage[cleveref]{csthm}
\end{verbatim}

Note that the \texttt{cleveref} option requires the \texttt{hyperref} package to be loaded.

\subsection{Theorem Environments}

The \texttt{csthm} package provides several theorem-like environments commonly used in computer science literature:

\begin{theorem}
Let \( G \) be a graph with \( n \) vertices. Then, the minimum number of colours needed to colour \( G \) such that no two adjacent vertices share the same colour is known as the chromatic number of \( G \).
\end{theorem}

\begin{lemma}
For every natural number \( n \), the sum of the first \( n \) odd numbers is \( n^2 \).
\end{lemma}

\begin{corollary}
The sum of the first \( n \) positive integers is given by \( \frac{n(n+1)}{2} \).
\end{corollary}

\begin{proposition}
If \( a \) and \( b \) are two even integers, then their sum is also even.
\end{proposition}

\begin{conjecture}
Every even integer greater than 2 can be expressed as the sum of two primes. (Goldbach's Conjecture)
\end{conjecture}

\subsection{Definition Environments}

To introduce key definitions and illustrative examples:

\begin{definition}
A \textit{tree} is a connected, undirected graph with no cycles.
\end{definition}

\begin{example}
Consider the binary tree with nodes labelled from 1 to 7. This tree has 3 levels, and each parent node has at most 2 children.
\end{example}

\subsection{Remark Environments}

To include remarks and notes that highlight important observations:

\begin{remark}
While all trees are graphs, not all graphs are trees. A graph must be acyclic and connected to be classified as a tree.
\end{remark}

\begin{note}
Keep in mind that proofs of conjectures, like Goldbach's Conjecture, often remain unproven for centuries despite numerous verified instances.
\end{note}

\subsection{Highlight Environments}

To emphasize crucial points within the document:

\begin{important}
Algorithm efficiency is critical; always consider time complexity when designing algorithms.
\end{important}

\begin{highlight}
Understanding the P vs NP problem is fundamental in computational complexity theory.
\end{highlight}

\subsection{Case Environment}

Used to present distinct cases in an argument or proof:

\begin{case}
\item When \( n = 0 \), the factorial of \( n \) is defined as 1.
\item When \( n > 0 \), the factorial is computed as \( n \times (n-1) \times \ldots \times 1 \).
\end{case}

\subsection{Axiom Environment}

To enumerate foundational axioms in formal proofs:

\begin{axiom}
\item For any sets \( A \) and \( B \), \( A \cup B = B \cup A \) (Commutative Law of Union).
\item \( A \cap (B \cup C) = (A \cap B) \cup (A \cap C) \) (Distributive Law).
\end{axiom}

\section{Customization}

You can customize the accent colour used in the package to suit your document's design preferences:

\begin{verbatim}
\setaccentcolor{blue}
\end{verbatim}

\section{License}

This package is released under the LaTeX Project Public License (LPPL) version 1.3c or later.

\section{Contact}

For bug reports or feature requests, please contact the package maintainer:

Agni Datta: \texttt{\href{mailto:agnidatta.org@gmail.com}{agnidatta.org@gmail.com}}

\section{Package Source Code}

The following listing shows the source code of the \texttt{csthm.sty} file:

\lstinputlisting{csthm.sty}

\end{document}
