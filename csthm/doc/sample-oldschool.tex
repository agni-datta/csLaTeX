\documentclass[9pt]{amsart}

% Microtypography and spacing
\usepackage[USenglish]{babel}
\usepackage{microtype}
\usepackage[onehalfspacing]{setspace}

% Geometry
\usepackage{geometry}
\geometry{a4paper,margin=1in}

% Essential packages
\usepackage[T1]{fontenc}
\usepackage{amsmath,amssymb,amsthm}

% Engine-specific packages (requires XeLaTeX or LuaLaTeX)
\usepackage{fontspec}
\usepackage{unicode-math}

% Font configuration
\usepackage[concrete]{fontsetup}
\setsansfont{Ysabeau}

% Color and hyperlinks
\usepackage[x11names,dvipsnames,svgnames]{xcolor}
\usepackage[colorlinks=true,linkcolor=RoyalBlue4]{hyperref}

% Nice theorem styles
\usepackage[oldschool]{csthm}

% Set a custom accent color
\setaccentcolor{RoyalBlue4}

\title{Demo of the \texttt{csthm} Package (OldSchool)}
\author{Agni Datta}

\begin{document}

\maketitle

\tableofcontents

\section{Introduction}

Welcome to the comprehensive demonstration of the \texttt{csthm} package! This
document serves as both a showcase and a practical guide for mathematical
writing in \LaTeX. The \texttt{csthm} package provides over 40 different
environments organized into four main categories, each serving specific
purposes in mathematical exposition.

The package offers three distinct visual styles: \texttt{oldschool} (used in
this demo), \texttt{normal}, and \texttt{fancy}. Each style is designed to
enhance readability while maintaining mathematical rigor. Throughout this
document, you'll see examples of every environment with detailed explanations
of their intended use cases.

Mathematical writing requires careful organization of ideas, and the right
environment can make the difference between a confusing exposition and a clear,
engaging presentation. Let's explore each category systematically.

\section{Theorem-like Environments}

Theorem-like environments are the backbone of mathematical exposition. They
present formal mathematical statements that have been proven or are accepted as
true. These environments typically appear in a distinctive style to emphasize
their importance and are usually numbered for easy reference.

\subsection{Core Theorem Environments}

The \texttt{theorem} environment is perhaps the most important in mathematical
writing. Use it for major results, fundamental principles, or any statement
that represents a significant mathematical truth. Theorems are typically the
culmination of mathematical reasoning and often serve as stepping stones to
more complex results.

\begin{theorem}[Fundamental Theorem of Arithmetic]
    Every integer greater than 1 is either prime or can be uniquely factored into prime numbers.
\end{theorem}

Notice how the theorem environment creates a visually distinct box with
automatic numbering. The optional argument in square brackets provides a name
or description, which is particularly useful for well-known results. This
theorem, for instance, is fundamental to number theory and deserves special
recognition.

Sometimes you may want to present a theorem without numbering, perhaps when
restating a result for emphasis or when the statement doesn't warrant a formal
number. The starred version accomplishes this:

\begin{theorem*}[Unnumbered Version]
    This is an unnumbered theorem for comparison. Use starred versions when you want the visual emphasis of a theorem environment without the formal numbering system.
\end{theorem*}

The \texttt{lemma} environment is perfect for auxiliary results that support
main theorems. Lemmas are often technical stepping stones that, while
important, are not the primary focus of your exposition. They're typically
proven before the main theorem and then used in that proof.

\begin{lemma}
    If $p$ is prime and $p | ab$, then $p | a$ or $p | b$.
\end{lemma}

This lemma about prime divisibility is a classic example. While it's a
significant result in its own right, it's often used as a tool to prove larger
theorems about prime factorization. The lemma environment signals to readers
that this is an important building block.

A \texttt{corollary} represents a direct consequence of a theorem or lemma.
These are results that follow easily from what has already been established.
Corollaries often reveal interesting implications of major theorems that might
not be immediately obvious.

\begin{corollary}
    There are infinitely many prime numbers.
\end{corollary}

This famous corollary follows from Euclid's proof technique and demonstrates
how major results often have elegant consequences. The corollary environment
helps readers understand the hierarchical structure of mathematical results.

The \texttt{proposition} environment is used for statements that are
significant but perhaps not as central as theorems. Propositions often present
interesting properties or intermediate results that are worth highlighting but
don't quite reach the level of a theorem.

\begin{proposition}
    The square root of 2 is irrational.
\end{proposition}

This classic proposition about $\sqrt{2}$ demonstrates a fundamental property
of real numbers. While extremely important, it's often presented as a
proposition rather than a theorem, depending on the context and emphasis of
your exposition.

Of course, formal mathematical statements require rigorous justification. The
\texttt{proof} environment provides the perfect framework for presenting
logical arguments. Proofs should be clear, complete, and well-structured.

\begin{proof}
    Assume $\sqrt{2} = \frac{p}{q}$ where $\gcd(p,q) = 1$. Then $2q^2 = p^2$, so $p^2$ is even, hence $p$ is even. Let $p = 2k$, then $2q^2 = 4k^2$, so $q^2 = 2k^2$. This means $q$ is also even, contradicting $\gcd(p,q) = 1$.
\end{proof}

Notice how the proof environment automatically includes a tombstone symbol
($\qedsymbol$) at the end, providing clear visual closure to the argument. The
proof demonstrates the classic technique of proof by contradiction.

\subsection{Extended Theorem Environments}

Beyond the core theorem environments, the \texttt{csthm} package provides
specialized environments for different types of mathematical statements. These
environments help you communicate the nature and certainty level of your
mathematical claims.

An \texttt{assertion} is used when you want to state something confidently but
perhaps without a complete formal proof. Assertions are particularly useful in
applied mathematics or when dealing with computational results.

\begin{assertion}
    Machine learning algorithms can approximate any continuous function with arbitrary precision given sufficient data and computational resources.
\end{assertion}

This assertion about machine learning capabilities illustrates how this
environment works well for statements that are generally accepted but might be
based on empirical evidence or informal arguments rather than rigorous
mathematical proof.

The \texttt{assumption} environment is crucial for setting up mathematical
contexts. Many mathematical results depend on specific conditions or
hypotheses, and the assumption environment makes these dependencies explicit.

\begin{assumption}
    The input data is independently and identically distributed, and the underlying distribution has finite variance.
\end{assumption}

Statistical and probabilistic results often depend heavily on assumptions about
data distribution. Making these assumptions explicit helps readers understand
the scope and limitations of subsequent results.

An \texttt{axiom} represents a fundamental principle that is accepted without
proof. Axioms form the foundation of mathematical systems and are typically few
in number but profound in their implications.

\begin{axiom}[Axiom of Choice]
    For any collection of non-empty sets, there exists a choice function that selects exactly one element from each set.
\end{axiom}

The Axiom of Choice is one of the most famous and controversial axioms in
mathematics. Its inclusion as an axiom rather than a theorem emphasizes its
foundational role in set theory and its acceptance as a basic principle.

A \texttt{claim} is useful for intermediate statements in proofs or for
assertions that you plan to justify later. Claims help break down complex
arguments into manageable pieces.

\begin{claim}
    The proposed sorting algorithm runs in $\mathcal{O}(n \log n)$ time in the worst case.
\end{claim}

Claims are particularly useful in algorithm analysis, where you might want to
state performance characteristics before providing the detailed analysis. The
claim environment signals that justification will follow.

The \texttt{conclusion} environment is perfect for summarizing results or
drawing final inferences from a series of arguments. Conclusions help readers
understand the ultimate significance of your mathematical exposition.

\begin{conclusion}
    Therefore, the proposed method is both computationally efficient and theoretically sound, making it suitable for large-scale applications.
\end{conclusion}

Conclusions often synthesize multiple results and highlight their practical
implications. This environment provides appropriate emphasis for such
synthesizing statements.

A \texttt{conjecture} represents an educated guess or hypothesis that hasn't
been proven but is believed to be true. Conjectures are among the most exciting
elements in mathematics, often driving research for decades or centuries.

\begin{conjecture}[Goldbach's Conjecture]
    Every even integer greater than 2 can be expressed as the sum of two primes.
\end{conjecture}

Goldbach's conjecture, despite extensive computational verification, remains
unproven. The conjecture environment appropriately emphasizes the speculative
nature of such statements while acknowledging their mathematical importance.

A \texttt{fact} is used for well-established results that are generally known
and don't require detailed proof in your context. Facts provide necessary
background information without interrupting the flow of your main argument.

\begin{fact}
    The complexity class P is contained in NP, and determining whether P = NP is one of the most important open problems in computer science.
\end{fact}

This fact about computational complexity is fundamental to theoretical computer
science. The fact environment is perfect for such widely-accepted background
information.

The \texttt{folklore} environment is uniquely useful for stating results or
principles that are widely known in the mathematical community but might not
have formal published proofs or definitive attributions.

\begin{folklore}
    ``Premature optimization is the root of all evil'' - a principle widely attributed to Donald Knuth, emphasizing that optimization should be guided by measurement rather than intuition.
\end{folklore}

Folklore results often represent accumulated wisdom in a field. While they
might not be formally proven, they represent important heuristics or principles
that practitioners have found valuable.

A \texttt{hypothesis} is essential for scientific mathematical work. Hypotheses
represent testable predictions or proposed explanations that can be
investigated through mathematical analysis or experimentation.

\begin{hypothesis}
    Increasing the training data size will improve model accuracy according to a logarithmic scaling law, with diminishing returns beyond a certain threshold.
\end{hypothesis}

Hypotheses in mathematical contexts often involve predictions about the
behavior of systems or algorithms. The hypothesis environment helps distinguish
these predictive statements from established results.

The \texttt{postulate} environment is used for fundamental assumptions that
form the basis of a mathematical theory. Postulates are similar to axioms but
are often more specific to particular mathematical systems.

\begin{postulate}[Relativity Postulate]
    All observers in inertial reference frames observe the same laws of physics, and the speed of light in a vacuum is constant for all observers.
\end{postulate}

Postulates often appear in mathematical physics or geometry, where they
establish the fundamental rules of a particular mathematical universe.
Einstein's postulates of special relativity exemplify how postulates can
revolutionize entire fields.

Finally, the \texttt{property} environment is perfect for highlighting specific
characteristics or attributes of mathematical objects. Properties help readers
understand the essential features of the objects under study.

\begin{property}[Associativity]
    For all elements $a, b, c$ in the group: $(a \cdot b) \cdot c = a \cdot (b \cdot c)$.
\end{property}

Group theory provides many examples where properties like associativity are
fundamental. The property environment helps emphasize these crucial
characteristics while maintaining clear organization.

\section{Definition-like Environments}

Definition-like environments are fundamental to mathematical exposition because
they establish the precise meaning of terms and concepts. Clear definitions are
essential for rigorous mathematical communication and help ensure that readers
understand exactly what is being discussed.

\subsection{Core Definition Environments}

The \texttt{definition} environment is the cornerstone of mathematical writing.
Every mathematical concept must be precisely defined, and the definition
environment provides the perfect framework for this essential task.

\begin{definition}[Graph]
    A graph $G = (V, E)$ consists of a finite set of vertices $V$ and a set of edges $E \subseteq V \times V$, where each edge connects two vertices.
\end{definition}

This definition of a graph demonstrates the typical structure: we name the
object being defined, provide its mathematical representation, and specify the
relationships between its components. The definition environment's distinctive
formatting helps readers immediately recognize that new terminology is being
introduced.

Sometimes you might want to present a definition without formal numbering,
perhaps when restating a definition for emphasis or when introducing informal
terminology:

\begin{definition*}[Unnumbered Definition]
    This shows how unnumbered definitions appear. Use this when you want the visual emphasis of a definition without adding to the formal numbering system.
\end{definition*}

The \texttt{notation} environment is specifically designed for introducing
mathematical symbols and notational conventions. Consistent notation is crucial
for mathematical clarity, and this environment helps establish and emphasize
notational choices.

\begin{notation}
    We denote the set of natural numbers by $\mathbb{N}$, the set of integers by $\mathbb{Z}$, the set of rational numbers by $\mathbb{Q}$, and the set of real numbers by $\mathbb{R}$.
\end{notation}

Mathematical notation can vary between sources, so explicitly stating your
notational conventions prevents confusion. The notation environment makes these
conventions prominent and easily referenced.

The \texttt{convention} environment establishes standard practices or
assumptions that will be used throughout your work. Conventions help streamline
exposition by establishing default assumptions.

\begin{convention}
    Throughout this paper, all graphs are assumed to be simple (no multiple edges or self-loops) and undirected unless explicitly stated otherwise.
\end{convention}

Conventions are particularly important in areas like graph theory, where there
are multiple standard definitions. By establishing conventions early, you can
avoid repetitive qualifications in later statements.

\subsection{Specialized Definition Environments}

The \texttt{construction} environment is perfect for presenting algorithmic or
step-by-step procedures for building mathematical objects. Constructions often
bridge the gap between abstract definitions and concrete implementations.

\begin{construction}[Binary Search Tree]
    To construct a binary search tree from a sequence of elements:
    \begin{enumerate}
        \item Start with an empty tree (root = null)
        \item For each element in the sequence:
              \begin{enumerate}
                  \item If the tree is empty, make the element the root
                  \item Otherwise, compare the element with the current node
                  \item If smaller, go to the left subtree; if larger, go to the right subtree
                  \item Repeat until finding an empty position and insert the element
              \end{enumerate}
        \item The resulting tree maintains the BST property: left subtree < node < right
              subtree
    \end{enumerate}
\end{construction}

This construction demonstrates how to build a fundamental data structure. The
construction environment is ideal for such procedural descriptions, making the
step-by-step nature clear and easy to follow.

The \texttt{problem} environment is essential for presenting mathematical
challenges or questions that require solution. Problems often motivate the
development of new techniques or illustrate the application of existing theory.

\begin{problem}[Traveling Salesman Problem]
Given a list of cities and the distances between each pair of cities, find the shortest possible route that visits each city exactly once and returns to the starting city.
\end{problem}

The TSP is a classic example of a computationally difficult problem that has
driven the development of optimization algorithms. The problem environment
helps emphasize the challenge and sets up the context for solution approaches.

A \texttt{protocol} is particularly useful in computer science and
cryptography, where step-by-step procedures involving multiple parties are
common. Protocols specify the exact sequence of actions required to achieve a
goal.

\begin{protocol}[Diffie-Hellman Key Exchange]
    To establish a shared secret key over an insecure channel:
    \begin{enumerate}
        \item Alice and Bob publicly agree on a large prime $p$ and a generator $g$ of the
              multiplicative group $\mathbb{Z}_p^*$
        \item Alice chooses a secret integer $a$ and sends $A = g^a \bmod p$ to Bob
        \item Bob chooses a secret integer $b$ and sends $B = g^b \bmod p$ to Alice
        \item Alice computes the shared secret: $s = B^a \bmod p = g^{ab} \bmod p$
        \item Bob computes the shared secret: $s = A^b \bmod p = g^{ab} \bmod p$
        \item Both parties now share the secret $s$ without ever transmitting it
    \end{enumerate}
\end{protocol}

This protocol demonstrates the power of public-key cryptography. The protocol
environment clearly delineates the steps and makes the interaction between
parties explicit.

The \texttt{application} environment highlights practical uses of mathematical
concepts. Applications help readers understand the relevance and impact of
abstract mathematical ideas.

\begin{application}
    Hash tables are widely used in database indexing, caching systems, and implementing associative arrays. Their average-case $\mathcal{O}(1)$ lookup time makes them indispensable for applications requiring fast data retrieval.
\end{application}

Applications bridge the gap between theory and practice, showing how
mathematical concepts solve real-world problems. This environment helps
emphasize the practical value of mathematical results.

The \texttt{experiment} environment is valuable for presenting empirical
investigations or computational studies. In an era of computational
mathematics, experiments often provide crucial insights.

\begin{experiment}
    We implemented the algorithm in Python and tested it on randomly generated datasets of varying sizes: 1K, 10K, 100K, and 1M records. Each test was repeated 100 times to ensure statistical reliability, and we measured both execution time and memory usage.
\end{experiment}

Computational experiments are increasingly important in mathematics and
computer science. The experiment environment helps distinguish empirical
investigations from theoretical analysis.

Finally, the \texttt{result} environment presents the outcomes of experiments
or investigations. Results often provide the empirical evidence needed to
support theoretical claims.

\begin{result}
    The proposed algorithm achieved 95.7\% accuracy on the test dataset, with a standard deviation of 1.2\%. Execution time scaled linearly with input size, confirming the theoretical $\mathcal{O}(n)$ complexity analysis.
\end{result}

Results provide concrete evidence for the performance or behavior of algorithms
and methods. The result environment helps distinguish these empirical findings
from theoretical predictions.

\section{Remark-like Environments}

Remark-like environments provide commentary, clarification, and additional
insights that enhance understanding without being part of the main logical
flow. These environments help create a more conversational and educational tone
in mathematical writing.

\subsection{Commentary and Clarification}

The \texttt{remark} environment is one of the most versatile tools in
mathematical exposition. Use it to provide additional insight, point out
special cases, or offer alternative perspectives on results.

\begin{remark}
    This result generalizes to higher dimensions with appropriate modifications. In $n$-dimensional space, the constant factor changes from $\pi$ to $\frac{\pi^{n/2}}{\Gamma(n/2 + 1)}$, but the fundamental structure of the proof remains unchanged.
\end{remark}

Remarks like this help readers understand the broader context and potential
extensions of results. They're perfect for insights that are valuable but not
essential to the main argument.

For informal comments that don't need numbering, use the starred version:

\begin{remark*}[Unnumbered Remark]
    Unnumbered remarks are useful for casual observations or side comments that enhance understanding but don't require formal reference. They maintain the visual emphasis while keeping the numbering system clean.
\end{remark*}

The \texttt{example} environment is crucial for illustrating abstract concepts
with concrete instances. Examples help readers understand definitions and
theorems by showing them in action.

\begin{example}
    Consider the function $f(x) = x^2$ on the interval $[0, 2]$. Its derivative is $f'(x) = 2x$, which is positive for all $x > 0$, confirming that $f$ is strictly increasing on $(0, 2]$. At $x = 1$, we have $f(1) = 1$ and $f'(1) = 2$, so the tangent line has equation $y = 2x - 1$.
\end{example}

This example demonstrates calculus concepts with specific calculations.
Examples should be carefully chosen to illuminate the most important aspects of
the concept being illustrated.

A \texttt{note} provides additional information or draws attention to important
details that might otherwise be overlooked. Notes are particularly useful for
pointing out connections between results.

\begin{note}
    The proof technique used here is essentially the same as the one employed in Theorem 2.3, but applied to a different metric space. This suggests that the result might generalize to a broader class of spaces.
\end{note}

Notes help readers recognize patterns and connections in mathematical
arguments. They're invaluable for building mathematical intuition and
understanding.

The \texttt{observation} environment is perfect for stating facts that are
noticed during analysis but aren't formally proven results. Observations often
lead to deeper insights or future research directions.

\begin{observation}
    The algorithm's performance appears to degrade linearly with input noise levels, suggesting that preprocessing to reduce noise could significantly improve results.
\end{observation}

Observations like this often come from experimental work or careful analysis of
examples. They help readers understand the practical behavior of mathematical
objects or algorithms.

\subsection{Educational and Explanatory Environments}

The \texttt{commentary} environment provides space for editorial remarks about
mathematical content. Commentary can discuss the historical development of
ideas, their significance, or their relationship to other work.

\begin{commentary}
    This approach represents a significant departure from traditional methods in numerical analysis. While classical techniques focus on polynomial approximation, this method leverages the geometric structure of the problem to achieve superior convergence rates.
\end{commentary}

Commentary helps place mathematical work in context and can provide valuable
perspective on the significance and novelty of results.

An \texttt{exercise} presents problems for readers to solve, promoting active
engagement with the material. Exercises should be carefully designed to
reinforce key concepts and develop problem-solving skills.

\begin{exercise}
    Prove that the sum of the first $n$ positive integers is $\frac{n(n+1)}{2}$ using mathematical induction. Then, use this result to find a formula for the sum of the first $n$ positive odd integers.
\end{exercise}

Exercises like this build mathematical skills progressively. The first part
reviews a fundamental technique, while the second part applies the result to a
related problem.

The \texttt{motivation} environment explains why certain concepts or approaches
are important. Motivation helps readers understand the purpose behind
mathematical development and maintains interest in the material.

\begin{motivation}
    Understanding the time complexity of recursive algorithms is crucial for several reasons: it helps predict performance on large inputs, guides optimization efforts, and provides insight into the inherent difficulty of computational problems.
\end{motivation}

Motivation is particularly important when introducing abstract concepts or
technical machinery. It helps readers understand why they should invest effort
in understanding the material.

The \texttt{notationabuse} environment acknowledges places where notation is
used loosely or where common shortcuts are employed. This promotes honesty in
mathematical communication.

\begin{notationabuse}
    We sometimes write $\mathcal{O}(f(n))$ when we technically mean $\Theta(f(n))$, following the common computer science convention where big-O notation is used to describe tight bounds rather than just upper bounds.
\end{notationabuse}

Acknowledging notation abuse helps prevent confusion and demonstrates awareness
of mathematical precision. It's particularly important in applied fields where
notation conventions may differ from pure mathematics.

Finally, the \texttt{question} environment poses inquiries that arise naturally
from the material. Questions can highlight open problems, suggest extensions,
or prompt readers to think more deeply about concepts.

\begin{question}
    Can this convergence result be extended to infinite-dimensional Banach spaces? What additional conditions would be necessary to ensure convergence in such spaces?
\end{question}

Questions like this encourage mathematical thinking and can point toward future
research directions. They help readers engage more actively with the material
and develop their own mathematical intuition.

\section{Highlight-like Environments}

Highlight-like environments are designed to draw special attention to crucial
information, warnings, insights, and key takeaways. These environments help
create visual hierarchy and ensure that important points aren't overlooked by
readers.

\subsection{Critical Information and Warnings}

The \texttt{important} environment is used to emphasize information that is
absolutely crucial for understanding or applying the material. Use this
environment sparingly to maintain its impact.

\begin{important}
    Always validate input data before processing to prevent security vulnerabilities, buffer overflows, and incorrect results. Failure to validate inputs is one of the most common sources of software bugs and security exploits.
\end{important}

Security considerations like this are genuinely important and deserve special
emphasis. The important environment ensures that such critical information
stands out from the surrounding text.

For critical notes that don't need numbering, use the starred version:

\begin{important*}[Critical Security Note]
    This is an unnumbered important note for emphasis. Use this when you want maximum visual impact without adding to the formal numbering system.
\end{important*}

The \texttt{warning} environment alerts readers to potential problems, common
mistakes, or dangerous practices. Warnings can prevent readers from making
costly errors or falling into common traps.

\begin{warning}
    Modifying the convergence parameter below 0.001 may cause numerical instability and lead to completely incorrect results. Always test parameter changes thoroughly before using them in production code.
\end{warning}

Warnings like this can save readers significant time and frustration. They're
particularly valuable in computational mathematics where parameter choices can
have dramatic effects on results.

A \texttt{tip} provides helpful advice or suggestions that can improve
understanding or performance. Tips often represent practical wisdom gained
through experience.

\begin{tip}
    Use version control for all your code, even small projects. Git repositories help track changes, enable collaboration, and provide backup. Initialize a repository with \texttt{git init} before writing any code.
\end{tip}

Tips like this provide practical guidance that goes beyond the immediate
mathematical content but supports effective mathematical work. They help
readers develop good practices.

\subsection{Insights and Key Points}

The \texttt{highlight} environment draws attention to particularly important or
interesting aspects of the material. Use highlights to emphasize key insights
or crucial connections.

\begin{highlight}
    The key insight is that the graph coloring problem can be reformulated as a constraint satisfaction problem, which opens up a wide range of algorithmic approaches including backtracking, forward checking, and arc consistency.
\end{highlight}

Highlights help readers identify the most important ideas in complex material.
They're particularly valuable for emphasizing conceptual breakthroughs or novel
approaches.

A \texttt{keypoint} emphasizes fundamental principles or essential facts that
readers should remember. Key points often summarize the most important aspects
of a section or concept.

\begin{keypoint}
    Performance optimization should always be based on actual profiling data rather than assumptions or intuition. The bottlenecks are often in unexpected places, and premature optimization can actually harm performance.
\end{keypoint}

Key points like this distill important principles that apply broadly. They help
readers extract the most valuable lessons from detailed technical material.

An \texttt{insight} presents a deeper understanding or novel perspective that
emerges from analysis. Insights often represent ``aha moments'' that
significantly advance understanding.

\begin{insight}
    The real bottleneck in this algorithm is not the computational complexity but the memory access patterns. Cache misses dominate execution time, which explains why the theoretical analysis doesn't match empirical performance.
\end{insight}

Insights like this often come from careful analysis or experimentation. They
represent genuine advances in understanding that can guide future work.

\subsection{Synthesis and Summary}

The \texttt{takeaway} environment summarizes the most important lessons or
conclusions from a section. Takeaways help readers extract the essential value
from complex material.

\begin{takeaway}
    Simplicity often leads to more maintainable and efficient solutions than clever optimizations. Focus on clear, correct implementations first, then optimize based on actual performance measurements.
\end{takeaway}

Takeaways like this distill practical wisdom that applies beyond the immediate
context. They help readers develop good judgment and effective approaches to
problem-solving.

A \texttt{summary} provides a concise overview of results, methods, or
conclusions. Summaries are particularly valuable at the end of sections or
after presenting multiple related results.

\begin{summary}
    We presented three sorting algorithms with different time complexities: bubble sort ($\mathcal{O}(n^2)$), merge sort ($\mathcal{O}(n \log n)$), and counting sort ($\mathcal{O}(n + k)$). The choice depends on input characteristics, stability requirements, and space constraints.
\end{summary}

Summaries help readers consolidate their understanding and see the big picture.
They're especially valuable in technical material where details might obscure
the main points.

The \texttt{recall} environment reminds readers of important facts or
principles that are relevant to the current discussion. Recall statements help
maintain context and prevent readers from forgetting crucial background.

\begin{recall}
    Remember that correlation does not imply causation. Even strong statistical relationships don't establish causal links without additional evidence such as controlled experiments or causal inference methods.
\end{recall}

Recall statements like this reinforce fundamental principles that are easy to
forget in the midst of technical details. They help maintain good mathematical
and scientific thinking.

Finally, the \texttt{guideline} environment presents best practices or
recommended approaches. Guidelines help readers develop good habits and avoid
common problems.

\begin{guideline}
    Follow the single responsibility principle: each function should have one clear, well-defined purpose. This makes code easier to understand, test, debug, and maintain.
\end{guideline}

Guidelines like this promote good programming and mathematical practices. They
help readers develop skills that will serve them well across many different
contexts.

\section{Case Environment Demo}

The \texttt{case} environment is a specialized tool for organizing proofs and
arguments that involve multiple scenarios or conditions. Case analysis is a
fundamental proof technique that helps break complex problems into manageable
parts.

Case analysis is particularly common in discrete mathematics, where problems
often involve considering different possibilities exhaustively. The case
environment provides clear visual organization for such arguments.

\begin{theorem}[Absolute Value Properties]
    For any real number $x$, we have $|x| \geq 0$, and $|x| = 0$ if and only if $x = 0$.
\end{theorem}

This theorem about absolute values naturally leads to a case-by-case analysis.
The proof demonstrates how the case environment organizes different scenarios
clearly:

\begin{proof}
    We prove both parts by considering all possible cases for the real number $x$:
    \begin{case}
        \item If $x \geq 0$, then by definition $|x| = x \geq 0$. Moreover, $|x| = 0$ if and
        only if $x = 0$.
        \item If $x < 0$, then by definition $|x| = -x > 0$ since $x$ is negative. In this
        case, $|x| \neq 0$ since $x \neq 0$.
    \end{case}
    In both cases, we have $|x| \geq 0$, and $|x| = 0$ occurs precisely when $x = 0$.
\end{proof}

Notice how the case environment creates a numbered list that clearly
distinguishes between different scenarios. This organization makes the proof
much easier to follow than a paragraph that tries to handle all cases
simultaneously.

Case analysis is also valuable in algorithm analysis, where different input
conditions can lead to different performance characteristics, and the case
environment is equally useful in mathematical proofs involving induction, where
base cases and inductive steps need clear organization.

\end{document}